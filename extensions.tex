\def\grule #1#2{
  \bTR \bTD #1 \eTD \bTD \useMPgraphic{arrow} \eTD \bTD #2 \eTD \eTR
}
\startuseMPgraphic{arrow}
drawarrow (0,0) .. (15,0)
\stopuseMPgraphic
\startcomponent[extensions]

\startchapter[title={Language Extensions}, reference={chapter:extensions}]
We extended the syntax of LOOP, WHILE and GOTO languages in a number of ways either to make their use
less tedious without altering their fundamental minimalistic nature or to allow the expression
of more complicated syntax in terms of already established functionality. The Theo-IDE accepts programs
written in the language \Important{Theo}, which is essentially the project languages, but combined
with the afformentioned extensions. \par
The main extensions are the following:
\startitemize
\item{multi-file programs using a \quotation{copy and paste} include directive}
\item{subroutines that can be used to represent common calculations}
\item{a powerful macro preprocessor that can be used to introduce custom syntax}
\stopitemize
We will now explain the differences between the project languages and Theo in more detail
by showing some examples. We expect that the reader is already quite familiar with the
use of the project languages. \par

\Important{Note}: In our examples we only make use of \Code{LOOP} syntax. \Code{WHILE} or \Code{GOTO}
syntax works just like in the project languages \cite[schoening] and may be used in the same way as shown in the
examples.

\startsection[title={Quality of Life}, reference={section:qol}]
As already mentioned, we introduced some simple quality-of-life changes that are
designed to make the use of Theo less tedious (but not less minimalistic) than
the project languages. \par
As an example take a look at the following program:
\startcommand
x0 := 1 ;
// this is a comment !
_coolVariable1 := x0
\stopcommand
The above program uses the following non-standard extensions:
\startitemize
\item{constant assignment: \Code{x0 := 1}}
\item{variable copying: \Code{_coolVariable1 := x0}}
\item{variable names that aren't $x_n$: \Code{_coolVariable1}}
\item{comments}
\stopitemize
As a result, the above program strictly speaking isn't written in the project languages, but
\Important{is} a valid Theo program.
\stopsection

\startsection[title={Subroutines}, reference={section:subroutines}]
Consider the following scenario: You have written a program that adds two variables together
(a basic excercise when starting to learn the project languages). Now you want to add two
variables together at multiple points in your program, but you don't want to copy the same
code snippet everywhere. Theo allows the definition of samller \Important{Programs} that you can
define in your code and execute on will. \par
A program definition may look like this for the above scenario:
\startcommand
PROGRAM add IN x0, x1 OUT x0 DO
  LOOP x1 DO
    x0 := x0 + 1
  END
END
\stopcommand
The identifiers after the \Code{IN}-keyword are the input variables.
The identifier listed after the \Code{OUT}-keyword is the variable
which is used as the result of the subroutine: When execution reaches
the \Code{END}-keyword of the program, the contents of this
variable are copied to the caller. \par
You may call subroutines utilizing the \Code{RUN}-keyword:
\startcommand
PROGRAM add IN x0, x1 OUT x0 DO
  LOOP x1 DO x0 := x0 + 1 END
END

arg := 4 ;
// x2 = arg + (arg + 5)
x2 := RUN add WITH
  arg,
  RUN add WITH arg, 5 END
END
\stopcommand
The values passed to a subroutine between \Code{WITH} and \Code{END}
are assigned to the identifiers of the \Code{IN}-section. All identifiers
used in a subroutine refer to values only accessible by the subroutine:
\startcommand
PROGRAM wrong IN x0 OUT x0 DO
  // b = 0, because all variables are initially 0
  b := a ;
  // this modifies the variable 'a' local to this scope
  a := 2
END
a := 1 ;
x0 := RUN wrong WITH 0 END
c := a
// c = 1, as the outer a isn't modified by the program call
\stopcommand
In other words, you can't access variables of outer scopes, there are always
only local variables. However you are able to call other programs as long
as they were defined before their use. \par
Programs are always defined at the top of your code (before the actual program starts),
and multiple program definitions are \Important{not} separated by semicolons:
\startcommand
PROGRAM add IN x0, x1 OUT x0 DO
  LOOP x1 DO x0 := x0 + 1 END
END

PROGRAM mul IN x0, x1 OUT x2 DO
  x2 := 0 ;
  LOOP x1 DO
    x2 := RUN add WITH x2, x0 END
  END
END

x1 := 4 ;
x2 := 5 ;
x0 := RUN mul WITH x1, x2 END
\stopcommand
\stopsection

\startsection[title={Programs Consisting of Multiple Files}, reference={section:multifile}]
Consider the following scenario where you have two or more files in
your Theo project. One named \quotation{math.theo}:
\startcommand
PROGRAM add IN x0, x1 OUT x0 DO
  LOOP x1 DO x0 := x0 + 1 END
END
\stopcommand
And another one, which is the main file of your program:
\startcommand
INCLUDE \"math.theo\"
x1 := 2 ; x2 := 3 ;
x0 := RUN add WITH x1, x2 END
\stopcommand

At the start of processing, the Theo compiler will insert the contents
of \quotation{math.theo} at the position where it is included. As a
consequence of this, you need to respect semicolon rules across file
boundaries (possibly requiring a semicolon at the end of a file or after
an include statement) and cyclic dependencies are strictly disallowed.
We recommend using the include functionality like in the example to
outsource program definitions. \par

Include statements may appear at any position in the source code.
\stopsection

\startsection[title={Macro Preprocessor}, reference={section:macros}]
Whilst subroutines endow the user with considerable power, they
are still limited. The compiler used in the Theo-IDE comes with a
powerful preprocessor with which you can define macro patterns
and replacements for levying a higher level of syntactic modification.
\par

Let's look at a simple example of macro usage. In the examples leading
up to this point, we have almost always defined a subroutine \Code{add},
which performs addition. We have then utilized this subroutine by writing
\Code{RUN add WITH value1, value2}, which is fine, but actually
writing something like \Code{value1 + value2} would have been way nicer.
\par

The following example demonstrates how to translate code snippets of the
form \Code{value1 + value2} to the more verbose subroutine call:
\startcommand
PROGRAM add IN x0, x1 OUT x0 DO
  LOOP x1 DO x0 := x0 + 1 END
END

DEFINE
  // we look for patterns of "value1 + value2"
  <VALUE> + <VALUE>
AS
  // and replace with this
  // ($0 is the contents of the first <VALUE>, $1 of the second)
  RUN add WITH $0, $1 END
END DEFINE

x0 := 5 ; x1 := 4 ;
// gets translated to RUN add WTIH x0, x1 END
x2 := x0 + x1 
\stopcommand
Macro definitions always start with \Code{DEFINE} and end with \Code{END DEFINE}.
Between these keywords, seperated by \Code{AS}, you will first find
the macro pattern that will be replaced and then the string that any
occurence of the pattern will be replaced with. \par

In the macro pattern definition, you may use special \Important{matcher tokens} which
don't match any token in particular, but a pre-defined class of tokens or strings of
tokens. The available matcher tokens are:
\startitemize
\item{\Code{<VALUE>}, which includes any identifier (variable name), constant or subroutine call}
\item{\Code{<ID>}, which matches any identifier}
\item{\Code{<INT>}, which matches any constant}
\item{\Code{<ARGS>}, which matches comma-separated identifiers (such as \Code{x0, x1, x2})}
\item{\Code{<PROGRAM>}, which matches entire semicolon-separated program blocks
  (such as \Code{x0 := 1 ; x1 := x0 + 1}), which includes any program
  according to the syntax rules of the project languages}
\stopitemize

With the powerful \Code{<PROGRAM>} matcher, you can for example
solve the common excercise of defining a \quotation{IF-THEN-ELSE}
statement whilst only using \Code{LOOP}:
\startcommand
DEFINE
  IF <VALUE> THEN <PROGRAM> ELSE <PROGRAM> END
AS
  #0 := $0 ;
  #1 := 0 ;
  #2 := 1 ;
  LOOP #0 DO
    #1 := 1 ;
    #2 := 0
  END ;
  LOOP #1 DO $1 END ;
  LOOP #2 DO $2 END
END DEFINE

x0 := 1 ;
IF x0 THEN
  x2 := 1
ELSE
  x2 := 2
END
\stopcommand

In the above example we made use of scratch variables
(variables that serve as temporary storage for a macro),
which are variables of the form \Code{\#n}, where n is any
natural number including zero.  \par

If you have multiple macro definitions, and want to
influence the order of replacement (which can matter
in some instances), you can define a \Important{replacement
  priority}:
\startcommand
... definitions of programs add and mul ...
DEFINE PRIO 10
  <VALUE> + <VALUE>
AS
  RUN add WITH $0, $1 END
END DEFINE

DEFINE PRIO 20
  <VALUE> * <VALUE>
AS
  RUN mul WITH $0, $1 END
END DEFINE
\stopcommand
If at any point multiple macros could be replaced
in the input, definitions with higher priority
will always be replaced before definitions with
lower priority. \par

In general, macros are replaced according to the
following schema: Replace the macro definition
with the highest priority, which was detected
at the topmost, leftmost position with the longest sequence
of tokens matched against it. Repeat this replacement
until nothing changes or an upper limit of allowed
replacements has been reached (this upper limit is
hard-coded in order to avoid infinite replacing
loops). \par

Macro definitions, just like \Code{include} directives, may
appear at any point in the source code and are \quotation{
  filtered out} at an early compilation step.

\stopsection

\startsection[title={Formal Summary}, reference={section:grammar}]
This section is intended only for completeness (and not for learning).
We will define a set of non-terminal symbols and grammars for
syntactically correct include directives, macro definitions and
transformed programs, which the informed reader may use to their
benefit.

\startsubsection[title={Non-Terminals}, reference={section:terms}]
\asin[table:tokens] lists all non-terminal symbols that the formal
grammars will use.
\startplacetable[title={Non-Terminal Symbols of Theo}, reference={table:tokens}, location=split]
\bTABLE[split=repeat]
  \setupTABLE[c][1][aligh=flushleft,style=\bold]
  \setupTABLE[c][2][style=\Code]
  \setupTABLE[c][3][align=flushleft]
  \setupTABLE[row][first][align={flushleft,lohi}]
\bTABLEhead
  \bTR
    \bTH Name \eTH 
    \bTH Regular Expression (FLEX)\eTH 
    \bTH Note \eTH 
  \eTR
\eTABLEhead
\bTABLEnext
  \bTR
    \bTH Name \eTH 
    \bTH Regular Expression (FLEX)\eTH 
    \bTH Note \eTH 
  \eTR
\eTABLEnext
\bTABLEbody  
  \bTR 
    \bTD id \eTD 
    \bTD \[a-zA-Z\_\]\[a-zA-Z0-9\_\]* \eTD 
    \bTD identifiers, variable names, program names \eTD 
  \eTR
  \bTR
    \bTD int \eTD
    \bTD (0\|(\[1-9\]\[0-9\]*)) \eTD
    \bTD constants \eTD
  \eTR
  \bTR
  \bTD file \eTD \bTD \letterbackslash\letterdoublequote\[\letterhat\letterbackslash\letterdoublequote\]*\letterbackslash\letterdoublequote \eTD \bTD file names \eTD
  \eTR
  \bTR
  \bTD insert \eTD \bTD \$\{\bold{int}\} \eTD \bTD matcher insertion point \eTD
  \eTR
  \bTR
  \bTD local \eTD \bTD \#\{\bold{int}\} \eTD \bTD macro scratch varialbe \eTD
  \eTR
  \bTR
  \bTD ws \eTD \bTD \[ \letterbackslash t\letterbackslash n\]+ \eTD \bTD whitespace, removed before grammar application \eTD
  \eTR
  \bTR
  \bTD comment \eTD \bTD \letterbackslash/\letterbackslash/.* \eTD \bTD comments, removed before grammar application \eTD
  \eTR
  \bTR
  \bTD , \eTD \bTD \letterbackslash , \eTD \bTD program argument separator \eTD
  \eTR
  \bTR
  \bTD ; \eTD \bTD \letterbackslash ; \eTD \bTD program statement separator \eTD
  \eTR
  \bTR
  \bTD : \eTD \bTD \letterbackslash : \eTD \bTD label declarator \eTD
  \eTR
  \bTR
  \bTD := \eTD \bTD \letterbackslash :\letterbackslash = \eTD \bTD assignment operator \eTD
  \eTR
  \bTR
  \bTD != 0 \eTD \bTD != 0 \eTD \bTD unequal zero operator \eTD
  \eTR
  \bTR
  \bTD = \eTD \bTD = \eTD \bTD equality operator \eTD
  \eTR
  \bTR
  \bTD run \eTD \bTD RUN\|run\|Run \eTD 
  \eTR
  \bTR
  \bTD with \eTD \bTD WITH\|with\|With \eTD
  \eTR
  \bTR
  \bTD do \eTD \bTD DO\|do\|Do \eTD 
  \eTR
  \bTR
  \bTD loop \eTD \bTD LOOP\|loop\|Loop \eTD
  \eTR
  \bTR
  \bTD while \eTD \bTD WHILE\|while\|While \eTD
  \eTR
  \bTR
  \bTD goto \eTD \bTD GOTO\|goto\|Goto \eTD
  \eTR
  \bTR
  \bTD if \eTD \bTD IF\|if\|If \eTD
  \eTR
  \bTR
  \bTD then \eTD \bTD THEN\|then\|Then \eTD
  \eTR
  \bTR
  \bTD stop \eTD \bTD STOP\|stop\|Stop \eTD
  \eTR
  \bTR
  \bTD end \eTD \bTD END\|end\|End \eTD
  \eTR
  \bTR
  \bTD program \eTD \bTD PROGRAM\|program\|Program\|\\PROG\|prog\|Prog \eTD
  \eTR
  \bTR
  \bTD in \eTD \bTD IN\|in\|In \eTD
  \eTR
  \bTR
  \bTD out \eTD \bTD OUT\|out\|Out \eTD
  \eTR
  \bTR
  \bTD include \eTD \bTD INCLUDE\|include\|Include \eTD
  \eTR
  \bTR
  \bTD define \eTD \bTD DEFINE\|define\|Define\|\\DEF\|def\|Def \eTD
  \eTR
  \bTR
  \bTD as \eTD \bTD AS\|as\|As \eTD
  \eTR
  \bTR
  \bTD priority \eTD \bTD PRIORITY\|priority\|Priority\|\\PRIO\|prio\|Prio \eTD
  \eTR
  \bTR
  \bTD enddefine \eTD \bTD END DEFINE\|end define\|\\End Define\|ENDDEF\|\\Enddef\|enddef \eTD
  \eTR
  \bTR
  \bTD <program> \eTD \bTD <PROGRAM>\|<program>\|\\<Program>\|<PROG>\|\\<prog>\|<Prog>\|<P>\|<p> \eTD \bTD macro program matcher \eTD
  \eTR
  \bTR
  \bTD <args> \eTD \bTD <ARGS>\|<args>\|<Args>\|<A>\|<a> \eTD \bTD macro argument chain matcher \eTD
  \eTR
  \bTR
  \bTD <value> \eTD \bTD <VALUE>\|<value>\|<Value>\|\\<VAL>\|<val>\|<Val>\|<V>\|<v> \eTD \bTD macro value matcher \eTD
  \eTR
  \bTR
  \bTD <int> \eTD \bTD <INT>\|<int>\|<Int> \eTD \bTD macro constant separator \eTD
  \eTR
  \bTR
  \bTD <id> \eTD \bTD <ID>\|<id> \eTD \bTD macro identifier matcher \eTD
  \eTR
  \bTR
  \bTD separator \eTD \bTD . \eTD \bTD any other sequence of characters (may be used for delimiters in macros, when you do not want \Code{<VALUE>} to treat it as an identifier  \eTD
  \eTR
\eTABLEbody
\eTABLE
\stopplacetable
\stopsubsection

\startsubsection[title={Include Directives}, reference={section:include-directives}]
\Code{include} statements are filtered out of the import at the first step of compilation.
Include directives start with \bold{include} and must be followed by \bold{file}.
\stopsubsection

\startsubsection[title={Macro Definitions}, reference={section:macro-defs}]
At the next step, macro definitions are extracted from the remaining input. The grammar in
\asin[fig:preproc-grammar], with starting symbol S, produces valid macro definitions. 

\startplacefigure[title={Grammar for macro definitions}, reference={fig:preproc-grammar}]
\bTABLE
\setupTABLE[row][each][topframe=off, bottomframe=off, leftframe=off, rightframe=off]
\bTR
\bTD S \eTD
\bTD \useMPgraphic{arrow} \eTD
\bTD \epsilon\space\| A S \| M S \eTD
\eTR
\bTR
\bTD A \eTD
\bTD \useMPgraphic{arrow} \eTD
\bTD \Important{any terminal symbol which is not \bold{define}, \bold{as}, \bold{enddefine}, \bold{insert}, \bold{local}, \bold{<program>}, \bold{<args>}, \bold{<id>}, \bold{<int>} or \bold{<value>}} \eTD
\eTR
\bTR
\bTD M \eTD
\bTD \useMPgraphic{arrow} \eTD
\bTD \bold{define} P D \bold{as} R \bold{enddefine} \eTD
\eTR
\bTR
\bTD P \eTD
\bTD \useMPgraphic{arrow} \eTD
\bTD \epsilon\space\| \bold{priority} \bold{int} \eTD
\eTR
\bTR
\bTD D \eTD
\bTD \useMPgraphic{arrow} \eTD
\bTD B \| B D \eTD
\eTR
\bTR
\bTD B \eTD
\bTD \useMPgraphic{arrow} \eTD
\bTD A \| \bold{<program>} \| \bold{<args>} \| \bold{<value>} \| \bold{<int>} \| \bold{<id>} \eTD
\eTR
\bTR
\bTD R \eTD
\bTD \useMPgraphic{arrow} \eTD
\bTD \epsilon\space\| C R \eTD
\eTR
\bTR
\bTD C \eTD
\bTD \useMPgraphic{arrow} \eTD
\bTD A \| \bold{local} \| \bold{insert}\eTD
\eTR
\eTABLE
\stopplacefigure
\stopsubsection

\startsubsection[title={Theo}, reference={section:theo}]
After the preprocessor is applied to the remaining input, the grammar
in \asin[fig:parser-grammar] is used to verify syntactic correctness.
\startplacefigure[title={Grammar of Theo}, reference={fig:parser-grammar}, width=\textwidth]
\setupTABLE[row][each][topframe=off, bottomframe=off, leftframe=off, rightframe=off]
\bTABLE
\grule{S}{def S \| P}
\grule{def}{\bold{program} \bold{id} input output \bold{do} P \bold{end}}
\grule{input}{\bold{in} L}
\grule{L}{\bold{id} \| L \bold{,} L}
\grule{output}{\epsilon\space\| \bold{out} \bold{id}}
\grule{P}{P \bold{;} P}
\grule{P}{A}
\grule{P}{\bold{id :} A}
\grule{A}{\bold{id := V}}
\grule{A}{\bold{loop id do} P \bold{end}}
\grule{A}{\bold{while id != 0 do} P \bold{end}}
\grule{A}{\bold{goto id} \| \bold{stop}}
\grule{A}{\bold{if id = int then goto id}}
\grule{V}{\bold{id} \| \bold{int} \| C}
\grule{C}{\bold{run id with} args \bold{end}}
\grule{args}{V \| args \bold{,} args}
\eTABLE
\stopplacefigure
\stopsubsection

\stopsection

\stopchapter

\stopcomponent
