\startcomponent[functionality]

\startchapter[title={Basic Functionality}, reference={chapter:functionality}]
Once the Theo-IDE application is opened, you will be greeted with the user interface
as pictured in \asin[fig:ui]. Depending on the window width of your platform,
the sidebar may be located at the bottom, and some of the buttons in the primary
action bar at the top may be hidden in drop-down menus. We will now proceed with
an explanation of the functionality of the Theo-IDE.

\startplacefigure[title={User Interface of Theo-IDE}, reference={fig:ui}]
\externalfigure[resources/theoide.png][width=\textwidth]
\stopplacefigure

The action bar at the top (which also contains the \quotation{Theo-IDE} label) is
refereed to as the \Important{primary action bar}. Below the primary action bar
the view is split into the \Important{editor} (on the left side in \asin[fig:ui])
and the \Important{sidebar}. The sidebar further is segmented into the
\Important{secondary action bar} at its top and the \Important{control area} below it.

\startsection[title={Creating, Opening and Saving Files}, reference={section:creating}]
You may create new program files by clicking on the button with the plus icon
in the primary action bar (the leftmost button in \asin[fig:ui]). In situations
where the window is narrow, this button may be hidden in a drop-down menu, which
is accessible from the primary action bar. \par
Upon clicking this button, a new editor tab named \Important{Temporary-*} will open,
and you'll be able to enter your source code.\par
You may load source code from files contained in your file system by clicking on the button
with the file icon in the primary action bar (next to the creating button in \asin[fig:ui]).
In narrow windows, this button is hidden in the same drop-down menu as the opening button.\par
Upon clicking this button, a file dialog will open, prompting you for the files you wish
to load. After confirming your selection of files, one editor tab will open for each
of the files you opened, where the name of the tab is the file name in your file system.\par
Analogously to opening files, you may save modified or created files by clicking on the button
with the save icon (to the right of the opening button in \asin[fig:ui]). \par
This action will save whatever file is currently selected in the editor.
\stopsection

\startsection[title={Running Programs}, reference={section:running}]
Running programs is a two-step process: First you select the main file of your program
in the control area by selecting the name of the editor tab that contains it from the
drop-down menu labeled \Important{Main Script}. Secondly, in the primary action bar,
you click on the button with the triangular icon. \par

This action will first compile your program. If any errors occur during compilation you
will be shown a message dialog notifying you of the incident. If no errors occur, your
program will be executed. When execution finishes successfully, a table showing the final
variable contents will appear in the control area under the label \Important{Output}.
Should your program not halt for any reason, you may use the stop button in the secondary
action bar (the one with the square iconography), to forcefully halt your program.
\stopsection

\startsection[title={Debugging Programs}, reference={section:debugging}]
If you want to step through your program line-by-line or have the program halt on
predefined breakpoints, you will want to make use of Theo-IDEs debugging
capabilities.\par
At first, you will follow exactly the same process as when running a program, except
that after selecting your main file, you will click on the debugging button (next to
the run button in \asin[fig:ui], with the insect icon).\par
At this point, your program will once again be compiled, and any errors will be reported
with a message dialog. But the program \Important{won't} be executed straight away, and
you will be able to make use of the buttons in the secondary action bar to influence
execution:
\startitemize
\item{
  the leftmost button in \asin[fig:ui] will execute the program until a breakpoint is encountered
  or the program ends (whichever happens first)
}
\item{
  the second button from the left in \asin[fig:ui] will execute one line of code
}
\item{
  the third button from the left in \asin[fig:ui] toggles auto-stepping mode: When enabled, Theo-IDE
  will automatically execute lines of code with a delay inbetween, the length of which is altered
  with the slider titled \Important{Auto Step Speed} in the control area
}
\item{
  the fourth button from the left in \asin[fig:ui] resets execution to the beginning of the program
}
\item{
  the fifth button from the left in \asin[fig:ui] stops execution
}
\stopitemize
You may activate or deactivate breakpoints by clicking on the line number in the editor at the location
where you want the program to pause. During execution, the output table in the control area will always
show the current variable states.

\stopsection

\stopchapter

\stopcomponent
