\startcomponent[introduction]

\startchapter[title={Introduction},reference={chapter:introduction}]
This document serves as an introduction to the use of Theo-IDE, a program developed for a project named
\Important{Integrated Development Enviroment for LOOP, WHILE and GOTO} at 
the University of Applied Sciences Würzburg-Schweinfurt.
LOOP, WHILE and GOTO \cite[schoening] (hereafter referred to as \Important{the project languages}) are the names of a range 
of simple concept programming languages used within lectures such as 
\Important{Foundations of Theoretical Computer Science} and are used for demonstrating
a range of concepts and isomorphisms in computability theory. Theo-IDE provides a simple explorative enviroment for 
developing, running and debugging programs written in these languages.

\startsection[title={Requirements}, reference={section:requirements}]
During the project, great emphasis was placed on accessability and portability. As a result, Theo-IDE is available on a range
of different desktop and mobile platforms. We officially support Linux (X11, Wayland), MacOS and Windows desktops and provide
an application package for Android (ARMv7, ARMv8). If you're willing to get your hands dirty yourself, the project has also been
verified to build for iOS.
\stopsection

\startsection[title={Setup}, reference={section:setup}]
Binary distributions for most major platforms are available in the project repository at
\goto{https://github.com/Theo-IDE/Theo-IDE/}[url(https://github.com/Theo-IDE/Theo-IDE/)].
If there happens to be no binary artifact for your platform, the repository also includes
a list of dependencies and build instructions.
\stopsection

\stopchapter
\stopcomponent
